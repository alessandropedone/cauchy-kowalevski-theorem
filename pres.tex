\documentclass{beamer}
\usetheme{Berlin}
\usecolortheme{seagull}
\usepackage[utf8]{inputenc}
\usepackage[T1]{fontenc}
\usepackage{csvsimple}
\usepackage{graphicx}
\usepackage{verbatim}
\usepackage{sansmathaccent}
\pdfmapfile{+sansmathaccent.map}

\graphicspath{ {./img/} } % Path relative to the main .tex file 

% per inserire immagini:
%\begin{figure}[H]
%\centering
%\includegraphics[scale=.8]{Images/EM on elliptic data %3.png}
%\caption{E-M on elliptic data, 3 clusters}
%\end{figure}


% STUTTURE FONDAMENTALI
% ipotesi
\newenvironment{ipotesi}%
{\quad\left|\quad\def\arraystretch{1.2}\begin{array}{@{}l@{}}}%
{\end{array}\right.}
% tesi
\newcommand{\tesi}[1]{\quad\left|\quad{#1}\right.}
% unico comando per ipotesi e tesi
\newcommand{\hpth}[2]
{
\begin{flalign*}
\quad\quad\quad
\text{Ipotesi}
&\begin{ipotesi}
#1
\end{ipotesi}&&\\
\quad\quad\quad
\text{Tesi}
&\tesi{#2}&&
\end{flalign*}
}
% quando ci sono 2 tesi distinte
\newcommand{\hpthth}[3]
{
\begin{flalign*}
\quad\quad\quad
\text{Ipotesi}
&\begin{ipotesi} 
#1
\end{ipotesi}&&\\
\quad\quad\quad
\text{Tesi 1}
&\tesi{#2}&&\\
\quad\quad\quad
\text{Tesi 2}
&\tesi{#3}&&
\end{flalign*}
}
% NOTAZIONE
% sistemi
\newenvironment{system}%
{\left\lbrace\begin{array}{@{}l@{}}}%
{\end{array}\right.}
% parte intera
\newcommand{\interior}[1]{\accentset{\circ}{#1}}
% norma
\newcommand\norm[1]{\left\lVert#1\right\rVert}
% absolute value
\newcommand\abs[1]{\left|#1\right|}


\title{Teorema di Cauchy-Kovalevskaya}
\author{Alessandro Pedone}
\institute{Politecnico di Milano}
\date{24 settembre 2024}

\begin{document}

\frame{\titlepage}
\begin{frame}
    \frametitle{Indice}
    \tableofcontents
\end{frame}


\section{Introduzione}

\begin{frame}
\frametitle{Sofya Vasilyevna Kovalevskaya (1850-1891)}
Diamo per nota la figura storica di Augustin-Louis Cauchy. 
Kovalevskaya è stata:
\begin{itemize}
\item una matematica russa allieva di Weierstrass
\item la \textbf{pima donna} a conseguire un dottorato (3 tesi risalenti al 1875) e a ottenere una cattedra in Europa (in matematica)
\end{itemize}
\end{frame}

\begin{frame}
Esistono diverse sue \textbf{rappresentazioni artistiche} sia in letteratura che nel cinema. Le più rilevanti sono:
\begin{itemize}
\item Una biografia accurata: Little Sparrow: A Portrait of Sophia Kovalevsky (1983), Don H. Kennedy
\item Un racconto breve: Too Much Happiness (2009), Alice Munro
\end{itemize}
\end{frame}


\begin{frame}
La domanda cruciale a cui vogliamo rispondere è la seguente: 
\begin{center}
\textit{E' possibile che esista una soluzione analitica \\ a un sistema di EDP qualsiasi?}
\end{center}
\end{frame}

\begin{frame}
La risposta a questa domanda sarà affermativa, per questo ci poniamo già delle altre: 
\begin{itemize}
\item sotto quali ipotesi?
\item la soluzione a questo sistema è unica?
\item la soluzione dipende in modo continuo dai dati?
\item quali conseguenze hanno risultati ottenuti?
\end{itemize}
\end{frame}

\begin{frame}
\frametitle{Superifci caratteristiche}
Prima di entrare nel merito della discussione è necessario introdurre il concetto di superficie caratteristica per un'equazione (strettamente legato al metodo delle caratteristiche).

Caso equazione lineare

Disegno
\end{frame}

\begin{frame}
Caso generale
\end{frame}

\begin{frame}
Significato (Evans): se assegnamo delle condizioni iniziali su una superficie non caratteristica sarà possibile calcolare tutte le derivate sulle superficie seguendo un ragionamento induttivo.
\end{frame}


\section{Versione invariante}

\begin{frame}
\frametitle{Background}
in quegli anni ci si concentrava sullo studio delle funzioni analitiche grazie al lavoro di Cauchy 1835-42 (analisi complessa), da qui segue quello di Kovalevskaya 70-74
l'esistenza e l'unicità di soluzioni locali (analitche/olomorfe) di equazioni differenziali ordinarie (che abbrevieremo con EDO da qui in poi) e di sistemi lineari del primo ordine, sfruttando il metodo dei maggioranti
\end{frame}

\begin{frame}
\frametitle{Metodo dei maggioranti}
Si basa sul teorema che garantisce che se $g_\alpha \geq |f_\alpha|$ e la serie con $g_\alpha$ come coefficienti converge allora lo fa anche la quella con coefficienti $f_\alpha$
\end{frame}

\begin{frame}
\frametitle{Schema dell'approccio}
Seguendo l'ordine cronologico di scoperta procediamo per generalizzazioni successive
\begin{itemize}
\item versione di base per EDO
\item versione per EDP quasi-lineari
\item generalizzazione per EDP non lineari
\end{itemize}
\end{frame}


\begin{frame}
\frametitle{EDO}
Teorema di unicità $f:A\times B\rightarrow\mathbb{C}^n$ olomorfa (A, B aperti)
\begin{equation}
\begin{system}
y' = f(x,y) \quad \forall x \in \Omega \\
y(x_0)=y_0
\end{system}
\end{equation}
Se la soluzione esiste è unica
\end{frame}

\begin{frame}
Teorema di esistenza locale con stima del raggio
\end{frame}

\begin{frame}
Sketch della dimostrazione
\end{frame}

\begin{frame}
\frametitle{EDP quasi-lineari}
Forma particolare del sistema di interesse
\end{frame}

\begin{frame}
DIMOSTRAZIONE: metodo delle caratteristiche e metodo dei maggioranti
\begin{enumerate}
\item si osserva come i coefficienti di una serie di potenze che risolve l'equazione devono essere dei polinomi a coefficienti non negativi
\item $A_i^*>>A_i, B^*>>B \implies u^*>>u$
\item si scelgono $A_i^*, B^*$ in modo tale da poter calcolare esplicitamente una soluzione analitica con il metodo delle caratteristiche
\end{enumerate}
\end{frame}

\begin{frame}
\frametitle{EDP non lineari}
Forma particolare del sistema di interesse
\end{frame}

\begin{frame}
\frametitle{EDP non lineari}
Trasformazione nel sistema quasi-lineare precedente (che è il grande merito di Kovalevskaya)
\end{frame}




\section{Esempi}

\begin{frame}
\frametitle{Esempio di Lewy}
Importanza della richiesta di analiticità
\end{frame}

\begin{frame}
generalizzazione esempio di Lewy, enunciato
\begin{theorem}
\hpth{
A \subseteq \mathbb{R}^3 \text{ aperto }\\
}
{
\exists \, F \in C^{\infty}(\mathbb{R}^3,\mathbb{R}) \; : \; \nexists \, u \in C^1(A,\mathbb{R}) \text{ tale che }
\begin{system}
Lu=F \text{ in } A\\
u_x,\,u_y,\,u_t \text{ soddisfano} \\
\text {la condizione di Hölder }
\end{system}
}
\end{theorem}
\end{frame}

\begin{frame}
Idea della dimostrazione:
\begin{enumerate}
\item
traslare il problema del teorema precedente in modo da ricondursi al caso di un generico punto $(x_0,y_0,t_0)$, usando come forzante la funzione $g(x,y,t)=f(t-2xy_0+2x_0y)$;
\item
costruire con una serie una funzione $S_a \in C^\infty$ per ogni $a \in l^\infty$;
\item
costruire degli insiemi $E_{j,n} \subseteq l^\infty$ chiusi e senza parte interna sfruttando $S_a$ e il teorema di Ascoli-Arzelà;
\item
concludere la dimostrazione del nuovo teorema utilizzando i lemmi appena citati per ricavare, con un ragionamento per assurdo, l'uguaglianza $l^\infty = \bigcup E_{j,n}$, grazie alla quale si può applicare l'argomento di Baire.
\end{enumerate}
\end{frame}

\begin{frame}
\frametitle{Esempio di Kovalevskaya}
Importanza superfici non caratteristiche
\end{frame}

\begin{frame}
\frametitle{Esempio di Hadamard}
Nessuna garanzia della stabilità della soluzione
\end{frame}



\section{Versioni alternative}

\begin{frame}
\frametitle{Versione classica}
Enunciato, può essere visto come corollario di un teorema più astratto.
\end{frame}

\begin{frame}
\frametitle{Versione astratta}
Premessa $$E_s = H(\overline{\mathcal{O}_s}; \mathbb{C}^m)$$ con $s \in [0,1]$, costante $C$
\end{frame}

\begin{frame}
Enunciato
\end{frame}

\begin{frame}
Dimostrazione esistenza
\end{frame}

\begin{frame}
Dimostrazione unicità
\end{frame}

\begin{frame}
\frametitle{Versioni "olomorfe"}
Si può rifare tutto con $t$ variabile complessa e i teoremi non cambiano. Lo stesso vale anche per la versione invariante normale.
\end{frame}




\section{Applicazioni}

\begin{frame}
Le conseguenze di questo teorema si osservano in vari campi, tra cui i principali sono:
\begin{itemize}
\item teoria delle equazioni differenziali
\item fisica matematica, dove ha fatto emergere numerose domande (cosa succede nella realtà quando esiste una soluzione analitica locale?)
\item geometria differenziale
\item teoria economica
\end{itemize}
\end{frame}

\begin{frame}
Impatto sulla teoria delle equazioni differenziali:
\begin{itemize}
\item confutare la congettura di Weierstrass (ogni funzione è 
\item teorema di Holmgren
\item Treves e Nierenberg per la ricerca di condizioni necessarie e/o sufficienti per l'esistenza di soluzioni locali
\item Hormander la teoria degli operatori differenziali lineari (con particolare attenzione alla condizioni necessarie)
\end{itemize}
\end{frame}

\begin{frame}
\frametitle{Teorema di Holmgren}
Enunciato astratto, si dimostra utilizzando la versione astratta di CK
\end{frame}


\begin{frame}
Enunciato concreto
\end{frame}

\begin{frame}
Sketch della dimostrazione
\end{frame}

\begin{frame}
\frametitle{Teorema di Cartan-Kahler}
Per quanto riguarda geometria differenziale e teoria economica abbiamo un risultato che seguire dal teorema di CK

Enunciato e applicazione al campo economico
\end{frame}

\begin{frame}
Nonostante la ricerca in questo ambito condotta in quegli anni
\begin{itemize}
\item non fosse guidata da applicazioni immediate
\item portò a risultati deludenti rispetto alle aspettative di Cauchy e Weierstrass
\end{itemize}
ha avuto un impatto gigantesco grazie alla comprensione delle soluzioni di sistemi di EDP che ci ha permesso di raggiungere.
\end{frame}

\begin{frame}

\begin{center}
\textit{
Era una vita -- gli costava dirlo, come ebbe ad ammettere,
perché si era sempre guardato dagli eccessivi entusiasmi --, era una vita
che aspettava di veder entrare nel suo studio un allievo del genere. 
Un allievo in grado di lanciargli una sfida assoluta, 
di non seguire soltanto il percorso spericolato della sua mente, 
ma se possibile di spiccare un volo più alto.
}
\end{center}
\null\hfill --- Alice Munro, \textit{Too Much Happiness}
\end{frame}

\end{document}
