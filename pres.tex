\documentclass{beamer}
\usetheme{Berlin}
\usecolortheme{seagull}
\usepackage[utf8]{inputenc}
\usepackage[T1]{fontenc}
\usepackage{csvsimple}
\usepackage{graphicx}
\usepackage{verbatim}
\usepackage{sansmathaccent}
\pdfmapfile{+sansmathaccent.map}

\graphicspath{ {./img/} } % Path relative to the main .tex file 

% per inserire immagini:
%\begin{figure}[H]
%\centering
%\includegraphics[scale=.8]{Images/EM on elliptic data %3.png}
%\caption{E-M on elliptic data, 3 clusters}
%\end{figure}


\title{Teorema di Cauchy-Kovalevskaya}
\author{Alessandro Pedone}
\institute{Politecnico di Milano}
\date{24 settembre 2024}

\begin{document}

\frame{\titlepage}
\begin{frame}
    \frametitle{Indice}
    \tableofcontents
\end{frame}


\section{Introduzione}

\begin{frame}
\frametitle{Sofya Vasilyevna Kovalevskaya (1850-1891)}
Diamo per nota la figura storica di Augustin-Louis Cauchy. 
Kovalevskaya è stata:
\begin{itemize}
\item una matematica russa allieva di Weierstrass
\item la \textbf{pima donna} a conseguire un dottorato (3 tesi risalenti al 1875) e a ottenere una cattedra in Europa (in matematica)
\end{itemize}
\end{frame}

\begin{frame}
Esistono diverse sue \textbf{rappresentazioni artistiche} sia in letteratura che nel cinema. Le più rilevanti sono:
\begin{itemize}
\item Una biografia accurata: Little Sparrow: A Portrait of Sophia Kovalevsky (1983), Don H. Kennedy
\item Un racconto breve: Too Much Happiness (2009), Alice Munro
\end{itemize}
\end{frame}


\begin{frame}
La domanda cruciale a cui vogliamo rispondere è la seguente: 
\begin{center}
\textit{E' possibile che esista una soluzione analitica \\ a un sistema di EDP qualsiasi?}
\end{center}
\end{frame}

\begin{frame}
La risposta a questa domanda sarà affermativa, per questo ci poniamo già delle altre: 
\begin{itemize}
\item sotto quali ipotesi?
\item la soluzione a questo sistema è unica?
\item la soluzione dipende in modo continuo dai dati?
\item quali conseguenze hanno risultati ottenuti?
\end{itemize}
\end{frame}

\begin{frame}
\frametitle{Superifci caratteristiche}
Prima di entrare nel merito della discussione è necessario introdurre il concetto di superficie caratteristica per un'equazione.

Caso equazione lineare

Disegno
\end{frame}

\begin{frame}
\frametitle{Superifci caratteristiche}
Caso generale
\end{frame}




\section{Versione invariante}

\begin{frame}
\frametitle{Background}
si parte dal lavoro di cauchy 1835-42, lavoro di kovalevskaya 70-74
l'esistenza e l'unicità di soluzioni locali (analitche/olomorfe) di equazioni differenziali ordinarie (che abbrevieremo con EDO da qui in poi) e di sistemi lineari del primo ordine, sfruttando il metodo dei maggioranti
\end{frame}

\begin{frame}
\frametitle{Schema dell'approccio}
\begin{itemize}
\item 
\item 
\end{itemize}
\end{frame}


\begin{frame}
\frametitle{ODE}
\end{frame}




\section{Esempi}

\begin{frame}
\frametitle{Esempio di Lewy}
Importanza della richiesta di analiticità
\end{frame}

\begin{frame}
generalizzazione esempio di Lewy
\begin{enumerate}
\item
traslare il problema del teorema \ref{Lewy} in modo da ricondursi al caso di un generico punto $(x_0,y_0,t_0)$, usando come forzante la funzione $g(x,y,t)=f(t-2xy_0+2x_0y)$;
\item
costruire una funzione $S_a \in C^\infty$ per ogni $a \in l^\infty$;
\item
costruire degli insiemi $E_{j,n} \subseteq l^\infty$ chiusi e senza parte interna sfruttando $S_a$ e il teorema di Ascoli-Arzelà;
\item
concludere la dimostrazione del teorema \ref{Lewy2} utilizzando i lemmi appena citati per ricavare, con un ragionamento per assurdo, l'uguaglianza $l^\infty = \bigcup E_{j,n}$, grazie alla quale si può applicare l'argomento di Baire.
\end{enumerate}
\end{frame}

\begin{frame}
\frametitle{Esempio di Kovalevskaya}
Importanza superfici non caratteristiche
\end{frame}

\begin{frame}
\frametitle{Esempio di Hadamard}
Nessuna garanzia della stabilità della soluzione
\end{frame}



\section{Versioni alternative}

\begin{frame}
\frametitle{Versione classica}
Enunciato, può essere visto come corollario di un teorema più astratto.
\end{frame}

\begin{frame}
\frametitle{Versione astratta}
Premessa $$E_s = H(\overline{\mathcal{O}_s}; \mathbb{C}^m)$$ con $s \in [0,1]$, costante $C$
\end{frame}

\begin{frame}
Enunciato
\end{frame}

\begin{frame}
Dimostrazione esistenza
\end{frame}

\begin{frame}
Dimostrazione unicità
\end{frame}

\begin{frame}
\frametitle{Versioni "olomorfe"}
Si può rifare tutto con $t$ variabile complessa e i teoremi non cambiano. Lo stesso vale anche per la versione invariante normale.
\end{frame}




\section{Applicazioni}

\begin{frame}
Le conseguenze di questo teorema si osservano in vari campi, tra cui i principali sono:
\begin{itemize}
\item teoria delle equazioni differenziali
\item fisica matematica, dove ha fatto emergere numerose domande (cosa succede nella realtà quando esiste una soluzione analitica locale?)
\item geometria differenziale
\item teoria economica
\end{itemize}
\end{frame}

\begin{frame}
Impatto sulla teoria delle equazioni differenziali:
\begin{itemize}
\item teorema di Holmgren
\item Treves e Nierenberg per la ricerca di condizioni necessarie e/o sufficienti per l'esistenza di soluzioni locali
\item Hormander la teoria degli operatori differenziali lineari (con particolare attenzione alla condizioni necessarie)
\end{itemize}
\end{frame}

\begin{frame}
\frametitle{Teorema di Holmgren}
Enunciato astratto, si dimostra utilizzando la versione astratta di CK
\end{frame}


\begin{frame}
Enunciato concreto
\end{frame}

\begin{frame}
Sketch della dimostrazione
\end{frame}

\begin{frame}
\frametitle{Teorema di Cartan-Kahler}
Per quanto riguarda geometria differenziale e teoria economica abbiamo un risultato che seguire dal teorema di CK

Enunciato e applicazione al campo economico
\end{frame}

\begin{frame}
Indagare cose che danno risultati deludenti e limitatamente applicabili non è inutile

Come funziona il processo della ricerca in matematica (che in un corso non si può fare)

Esempio principe della matematica al femminile
\end{frame}

\end{document}