\documentclass[12pt]{report}

\usepackage{amsmath,amsfonts,amssymb}
\usepackage{graphicx}
\usepackage[left=2cm,right=2cm,top=2cm,bottom=2cm]{geometry}

\usepackage{float} % per il comando [H] per le tabelle
\usepackage{enumerate} % per scegliere i caratteri degli elechi

\author{Alessandro Pedone}
\title{Sull'esistenza di soluzioni locali\\ di equazioni differenziali alle derivate pariziali}
\date{\today}

\begin{document}

\maketitle

\chapter*{Prefazione}
Prerequisiti\\
Struttura dei capitoli e senso della trattazione: dare uno sguardo di insieme ai teoremi più importanti per la teoria locale, nelle forme più generali possibile, per poi dare un focus successivo alle equazioni lineari dove si riescono ad enunciare con efficacia condizioni sufficienti (Holmgren) e condizioni necessarie (Hörmander).\\
NOTAZIONE MULTI-INDICE\\
Derivate parziali
$$D^\alpha(u)$$
$$D^{ke_i}(u)=\frac{\partial ^ku}{\partial x_i^k}=\partial ^k_{x_i}u$$
$$k=1 \implies \partial _{x_i}u=u_{x_i}$$
Gradiente
$$Du=\nabla u=\left[u_{x_1},\ldots,u_{x_n}\right]$$
Matrice Hessiana
$$D^2u=Hu=\left[
\begin{matrix}
\nabla u_{x_1}\\
\vdots\\
\nabla u_{x_n}
\end{matrix}
\right]$$

Proprietà fondamentali
Leibniz...

\tableofcontents
\include{capitoli/1-nozioni-propedeutiche}
\chapter{Teorema di Cauchy-Kovalevski}
Perché non Kovalevskaya
Chi è K
Cronologia delle dimostrazioni
Abbreviato CK
\section{Versione per EDO}
Versione per EDO
\section{Versione per EDP quasi-lineari}
Versione per EDP quasi-lineari
\section{Versione per EDP non lineari}
Versione per EDP non lineari
Riscrivere l'equazione come un problema di evoluzione (vedi pdf)
\section{Versione astratta}
Versione astratta
\section{Esempi}

I prossimi tre esempi riguardano:
\begin{enumerate}[1]
\item Importanza della richiesta di analiticità
\item Importanza delle superfici non caratteristiche
\item Instabilità delle soluzioni di equazioni ellittiche
\end{enumerate}

\setcounter{example}{0}


\begin{example}[\textbf{Lewy}]

\end{example}



\begin{example}[\textbf{Kovalevski}]
Questo esempio, dovuto a Kovalevski stessa, è utile a comprendere più a fondo, in modo quanto più essenziale possibile, 
l'importanza, o meglio la necessità, di assumere che la superficie scelta per assegnare i dati di Cauchy non sia caratteristica 
per l'equazione differenziale in osservazione.\\ 
Si consideri quindi il seguente problema di Cauchy per l'equazione del calore in una dimensione:
\begin{align} 
\label{eq:1}
u_t-u_{xx}&=0\\ 
\label{eq:2}
u(x,0)&=\frac{1}{1+x^2} \quad \forall x \in \mathbb{R}
\end{align}
\begin{remark}
La condizione per $u_x$ su $\Gamma$ necessaria per completare il problema di Cauchy è già implicitamente imposta dall'equazione (\ref{eq:2}).
\end{remark}
L'obiettivo che ci si pone è quello di dimostrare che non ammette soluzioni analitiche in un intorno dell'origine.

\begin{enumerate}
\item
Per cominciare si osserva che in questo caso la superficie su cui sono stati assegnati i dati di Cauchy $(1.2)$ è $\Gamma:=\left\lbrace(x,t) \in \mathbb{R}^2:t=0\right\rbrace$. Essa in ogni punto ha come versore normale $(0,1)$ ed è, quindi, caratteristica per l'equazione (\ref{eq:1}), poiché
$$\sum_{|\alpha|=2}^{\;} a_\alpha \boldsymbol{\nu}^\alpha = a_{(2,0)}\boldsymbol{\nu}^{(2,0)} = 0.$$
\item
Per assurdo si supponga di avere una soluzione del problema $u$ analitica in un intorno dell'origine, ovvero:
$$u(x,t) = \sum_{\alpha = (\alpha_1, \alpha_2) }^{\;} c(\alpha) x^{\alpha_1} t ^{\alpha_2}, \quad
 c(\alpha) = \frac{D^\alpha u(0,0)}{\alpha!}$$
dove $|(x,t)|<r$ per qualche $r>0$.
\item
Si calcola i valori dei coefficienti $c(2n,0)$ $\forall n \in \mathbb{N}$.\\
Per fare questo si sviluppa in serie di potenze centrata nell'origine la funzione del problema di Cauchy:
$$\frac{1}{1+x^2} = \frac{d}{dx}\arctan(x) = \frac{d}{dx}\sum_{n=0}^{\infty}\frac{(-1)^n}{2n+1}x^{2n+1} 
= \sum_{n=0}^{\infty}(-1)^n x^{2n} \quad \forall x \in \mathbb{R}.$$
Da questa serie si ottengono le seguenti relazioni:
\begin{align*}
D_x^{2n}u(0,0) &= \frac{d^{2n}}{dx^{2n}} \left. \frac{1}{1+x^2}\right|_{x=0} = (-1)^n (2n)!\\
D_x^{2n+1}u(0,0) &= \frac{d^{2n+1}}{dx^{2n+1}} \left. \frac{1}{1+x^2} \right|_{x=0} = 0
\end{align*} 
dalle quali si ricava: $c(2n,0)=(-1)^n$ e $c(2n+1,0)=0$.
\item
Si calcolano i valori dei coefficienti $c(2n,n)$ e si dimostra che  $c(2n,n) \xrightarrow{n\rightarrow\infty} +\infty$.\\
Per fare questo, invece, si sfrutta l'equazione $(1.1)$ per ottenere la seguente relazione tra i coefficienti:
\begin{equation} 
\label{eq:3}
c(\alpha_1,\alpha_2+1) = \frac{(\alpha_1+2)(\alpha_1+1)}{(\alpha_2+1)}c(\alpha_1+2,\alpha_2).
\end{equation}
Si utilizza, quindi, quest'ultima 
\begin{align*}
c(2n,n) &= \frac{(2n+2)(2n+1)}{n}c(2n+2,n-1)   &(\ref{eq:3})\quad\text{con}\quad\alpha_1=2n,\; \alpha_2+1=n\\
 &= \ldots = \frac{(2n+2n)\cdots(2n+1)}{n!}c(2n+2n,0) &\text{\quad iterando su } n\\
 &= \frac{(4n)!}{(2n)!n!} (-1)^{2n} \\
 &\sim \frac{1}{\sqrt{\pi n}}\left(\frac{64n}{e}\right)^n \xrightarrow{n\rightarrow\infty} +\infty  &\text{per la formula di Stirling}
\end{align*}
\item
Si completa il ragionamento in modo immediato osservando che 
$c(2n,n) x^{2n} t ^{n}\xrightarrow{n\rightarrow\infty} +\infty$ $\forall (x,t) \neq (0,0)$, 
infatti ciò implica direttamente che la serie di potenze non converge in ogni punto diverso dall'origine e questo è assurdo.
\end{enumerate}
\end{example}


\begin{example}[\textbf{Equazione di Laplace}]

\end{example}


\chapter{Soluzioni locali di equazioni lineari}

\section{Versione astratta del teorema di Holmgren}
Versione astratta del teorema di Holmgren
\section{Versione classica del teorema di Holmgren}
Versione classica del teorema di Holmgren
\section{Condizioni di Hörmander}\label{Hormander}
Condizioni necessarie per l'esistenza locale di soluzioni per equazioni lineari\\
Hormander\\
On local solvability di Treves
\begin{itemize}
\item
sottolinea come i risultati negativi nella teoria locale siano molto rilevanti 
in quanto hanno implicazioni globali
\item
Le condizioni necessarie di Hormander hanno ispirato il lavoro di Treves e Nirenberg per la ricerca 
di condizioni necessarie e sufficienti
\end{itemize}

pag57-58 Folland


\appendix
\chapter{Notazione multi-indice}
esempio
\include{appendici/B-varietà}
\chapter{Serie di Taylor}
esempio
\chapter{Operatore di Cauchy-Riemann}
esempio
\include{appendici/E-analisi-funzionale}

\end{document}