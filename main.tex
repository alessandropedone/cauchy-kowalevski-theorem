\documentclass[12pt,twoside]{report}

% PACCHETTI FONDAMENTALI
\usepackage{amsmath,amsfonts,amssymb,amsthm, dsfont, mathtools}
\usepackage{graphicx}
\usepackage[left=2cm,right=2cm,top=3.3cm,bottom=2cm]{geometry}
\usepackage{float} % per il comando [H] per le tabelle
\usepackage{enumerate} % per scegliere i caratteri degli elechi
\usepackage{accents}
\usepackage{esint}
% intestazione pagine
\usepackage{fancyhdr}

% NOMENCLATURA
% indice
\renewcommand*\contentsname{Indice}
% capitoli
\renewcommand{\chaptername}{Capitolo}
% appendici
\renewcommand{\appendixname}{Appendice}
% bibliografia
\renewcommand\bibname{Bibliografia}
    
% TITOLI
% per mettere i titoli dei capitoli sulla destra e aggiungere una riga di separazione sotto
\usepackage{titlesec} 
\newcommand*{\justifyheading}{\raggedleft}
\titleformat{\chapter}[display]
  {\normalfont\LARGE\bfseries\justifyheading}
  {\chaptertitlename\ \thechapter \vspace*{-0.5cm}}
  {20pt}{\Huge}
  [\vspace*{0.3cm}\hrule height 0.08cm \vspace*{1cm}]

% STUTTURE FONDAMENTALI
% teoremi
\theoremstyle{plain}
\newtheorem{theorem}{Teorema}[section]
% lemmi
\newtheorem{lemma}[theorem]{Lemma}
% teoremi con nomi
\newtheoremstyle{named}{}{}{\itshape}{}{\bfseries}{.}{.5em}{\thmnote{#3} #2}
\theoremstyle{named}
\newtheorem{namedtheorem}[theorem]{Teorema}
% ipotesi
\newenvironment{ipotesi}%
{\quad\left|\quad\def\arraystretch{1.2}\begin{array}{@{}l@{}}}%
{\end{array}\right.}
% tesi
\newcommand{\tesi}[1]{\quad\left|\quad{#1}\right.}
% unico comando per ipotesi e tesi
\newcommand{\hpth}[2]
{
\begin{flalign*}
\quad\quad\quad
\text{Ipotesi}
&\begin{ipotesi}
#1
\end{ipotesi}&&\\
\quad\quad\quad
\text{Tesi}
&\tesi{#2}&&
\end{flalign*}
}
% quando ci sono 2 tesi distinte
\newcommand{\hpthth}[3]
{
\begin{flalign*}
\quad\quad\quad
\text{Ipotesi}
&\begin{ipotesi} 
#1
\end{ipotesi}&&\\
\quad\quad\quad
\text{Tesi 1}
&\tesi{#2}&&\\
\quad\quad\quad
\text{Tesi 2}
&\tesi{#3}&&
\end{flalign*}
}
% dimostrazioni
\renewcommand*{\proofname}{\bf{Dimostrazione:}}
\renewcommand\qedsymbol{\textsc{qed}}
% definizioni
\theoremstyle{definition}
\newtheorem{definition}{Definzione}[section]
% esempi
\newtheorem{example}{Esempio}
% osservazioni
\theoremstyle{remark}
\newtheorem*{remark}{Osservazione}

% NOTAZIONE
% sistemi
\newenvironment{system}%
{\left\lbrace\begin{array}{@{}l@{}}}%
{\end{array}\right.}
% parte intera
\newcommand{\interior}[1]{\accentset{\circ}{#1}}
% norma
\newcommand\norm[1]{\left\lVert#1\right\rVert}
% absolute value
\newcommand\abs[1]{\left|#1\right|}

% PAGINA BIANCA
\usepackage{afterpage}
\newcommand\blankpage{%
    \null
    \thispagestyle{empty}%
    \newpage}

% ALTRI
% indentazione del testo a 0
\parindent 0px
% numerazione in align*
\newcommand\numberthis{\addtocounter{equation}{1}\tag{\theequation}}
% citazione
\usepackage{epigraph}




\begin{document}
%
% non contare la pagina del titolo
\pagenumbering{gobble}

% titolo
\thispagestyle{empty}

\vspace*{-2.5cm} 
\mdseries{

\begin{center}
\includegraphics[width=5cm]{logo.png}

\vspace*{0.6cm}
{\Large\textsc{Politecnico di Milano}}\\
\rule{7cm}{1pt}

\vspace*{0.5cm}

Corso di Laurea Triennale in \textsc{Ingegneria Matematica}\\
Scuola di \textsc{Ingegneria Industriale e dell'Informazione}\\
\vspace*{1.3cm} 
{\LARGE\textmd{\textbf{
Il teorema di Cauchy-Kowalevski \\ e le sue conseguenze
}}}

\vspace*{1.5truecm} 

{\small Tesi di}
{\large\vspace*{0.3cm}\\Alessandro Pedone}

\vspace*{1.3cm}

\begin{tabular}{@{}ll}
\small
Relatore:\\[0.5cm]
\normalsize
\quad Prof. Maurizio Grasselli & .......................................\\[1cm]
\small
Candidato:\\[0.5cm]
\normalsize
\quad Alessandro Pedone & .......................................\\
\end{tabular}
\vfill
\rule{6cm}{1pt}

\small
Sessione di Laurea Settembre 2024\\
Anno Accademico 2023/2024
\end{center} 
\clearpage
}
\blankpage

% conta con i numeri romani 
\pagenumbering{roman}

% citazione
\setlength\epigraphwidth{8cm}
\setlength\epigraphrule{0pt}
\vspace*{\fill}
\epigraph{\textit{All his life -- he had difficulty saying this, as he admitted, being always wary of too much enthusiasm -- all his life he had been waiting for such a student to come into this room. A student who would challenge him completely, who was not only capable of following the strivings of his own mind but perhaps of flying beyond them.}}{--- \textup{Alice Munro}, \textit{Too Much Happiness}}
\vspace*{\fill}
%indice
\tableofcontents

\newpage
\blankpage
%\include{capitoli/sommario}

% ricomincia a contare con i numeri arabi
\newpage
\pagenumbering{arabic}

% rimuovi i numeri a piè di pagina
\makeatletter
\let\ps@plain\ps@empty
\makeatother

% inserisci le intestazioni per i capitoli
\pagestyle{fancy}
\renewcommand{\chaptermark}[1]{\markboth{\textit{\thechapter.\ #1}}{}}
\renewcommand{\sectionmark}[1]{\markright{\textit{\thesection.\ #1}}}
\fancyhead{} % cancella tutti i campi
\fancyhead[RO,LE]{\bfseries \thepage}
\renewcommand{\headrulewidth}{0.4pt}
\cfoot{}
\fancyhead[LO]{\rightmark}
\fancyhead[RE]{\leftmark}
\setlength{\headheight}{18pt}

%\include{capitoli/introduzione}
%\include{capitoli/versione-invariante}
\include{capitoli/esempi}
%\include{capitoli/versioni-c-a}
%\include{capitoli/holmgren}

% BIBLIOGRAFIA
% PER FARLA FUNZIONARE: eseguire latex-bibtex-latez-latex
\addcontentsline{toc}{chapter}{Bibliografia}
\nocite{*} % Inserire nella bibliografia anche le fonti non citate esplicitamente nel testo
\bibliographystyle{alpha} % We choose the "plain" reference style
\bibliography{bibliografia} % Entries are in the refs.bib file

\end{document}