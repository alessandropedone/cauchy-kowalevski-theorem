\chapter{Introduzione.}
Abbreviato CK

\section{Chi era Kowalevski?}

Sofya Vasilyevna Kovalevskaya

\textbf{Perché non Kovalevskaya}:  La stessa Kovalevskaya, per le pubblicazioni accademiche internazionali, soleva firmarsi Sophie Kowalevski

\textbf{Biografia}: background familiare, studi in germania, aiuto di Weierstrass (lezioni private e quattro tesi di dottorato), idee politiche (socialiste e radicali, sua sorella Anna le diede copies of radical journals of the time discussing Russian nihilism) e movimenti femministi, successo grazie alle tesi (I suoi risultati, tra cui il Teorema di Cauchy-Kovalevskaya, furono pubblicati nel 1875. Fu così che ottenne, prima donna in Europa, un dottorato in matematica), ritorno in russia inutile per la sua carriera accademica, in svezia dopo la morte del marito (Divenne, prima donna al mondo, professoressa di matematica, ottenendo la cattedra all'Università di Stoccolma (Högskola)), anche produzione letteraria, morte prematura a 41 anni nel 1891 di polmonite

\textbf{Nichilismo antico}: For the nihilists, science appeared to be the most effective means of helping the mass of people to a better life. Science pushed back the barriers of religion and superstition, and "proved" through the theory of evolution that (peaceful) social revolutions were the way of nature. For the early nihilists, science was virtually synonymous with truth, progress and radicalism; thus, the pursuit of a scientific career was viewed in no way as a hindrance to social activism. In fact, it was seen as a positive boost to progressive forces, an active blow against backwardness.

\textbf{Contributi}: nell'equazioni differenziali alle derivate parziali (teorema di CK) e nella meccaninca (Lagrange, Euler, and Kovalevskaya tops)

\textbf{Rappresentazione cinematografica}

Ayan Gasanovna Shakhmaliyeva è nata il 12 novembre 1932. Luogo di nascita: Baku, Azerbaijan SSR, USSR [ora Azerbaijan]. È conosciuta come regista e aiuto regista. È celebre per aver partecipato a Eto bylo u morya ... (1989), Malchishki (1970) e Dom naprotiv (1958). Morì il 27 aprile 1999. “Sofya Kovalevskaya” (1985, 3 episodi, 218 minuti, biografico, Lenfilm, OTF, colore). Gran Premio al Festival Internazionale del Film Televisivo Multiepisodio di Pianciano Terme, Italia, nel 1985.

A Hill on the Dark Side of the Moon (Swedish: Berget på månens baksida) is a 1983 Swedish drama film directed by Lennart Hjulström

\textbf{In letteratura}

\textbf{Cronologia delle dimostrazioni}
\begin{itemize}
\item
L'anno dopo la sua morte, la sua cara amica Anne Charlotte Leffler, sorella del matematico Gösta Mittag-Leffler e moglie dell'algebrista italiano Pasquale del Pezzo, le dedicò una biografia (Sonja Kovalevsky. Ciò che ho vissuto con lei e ciò che mi ha detto di sé, Ed. Albert Bonniers, Stoccolma, 1892)
\item 
Little Sparrow: A Portrait of Sophia Kovalevsky (1983), Don H. Kennedy, Ohio University Press, Athens, Ohio
\item
Beyond the Limit: The Dream of Sofya Kovalevskaya (2002) a biographical novel by mathematician and educator Joan Spicci, published by Tom Doherty Associates, LLC
\item
"Too Much Happiness" (2009), short story by Alice Munro, published in the August 2009 issue of Harper's Magazine (ispirato dal primo, infatti il racconto ripercorre gli ultimi giorni di vita di Sof'ja Kovalevskaja arricchito da reminiscenze del passato che Munro ha acquisito da lettere, diari e scritti. Munro ha potuto accedere a tali documenti tramite la moglie di Don H. Kennedy la quale è una lontana discendente di Kovalevskaja)
\end{itemize}


\chapter{Nozioni propedeutiche}
In questo capitolo vengono trattate delle nozioni fondamentali per la trattazione della teoria locale presentata.\\
Si tratta il metodo delle caratteristiche che sarà cruciale nella dimostrazione del teorema di CK anche se in una forma meno generale di quella presentata

\section{Equazioni differenziali alle derivate parziali}
\section{Superfici caratteristiche}
Cosa si intende per superficie analitica\\
Definizione di superficie caratteristica

\begin{enumerate}[i.]
\item
caso $t=0$ e calcolo di tutte le derivate (evans)
\item
caso lineare (folland)
\item
caso quasi lineare (folland)
\item
caso generale e calcolo di tutte le derivate (evans)
\item 
classificazione delle EDP
\end{enumerate}


\section{Metodo delle caratteristiche}

\begin{enumerate}[i.]
\item
caso lineare (folland)
\item
caso quasi lineare (folland)
\item
caso generale (evans)
\end{enumerate}
