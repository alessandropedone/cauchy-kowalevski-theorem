\chapter{Il teorema di Cauchy-Kovalevski}
\section{Versione per EDO}
Versione per EDO
\section{Versione per EDP quasi-lineari}
Versione per EDP quasi-lineari
Rifacendoci a Evans, e quindi anche usando la notazione in esso presente, 
assumiamo che i coefficienti del sistema $B_j$ e $c$ abbiano come raggi di convergenza $r_{B_j}>0$ e $r_c>0$ 
di conseguenza per il Lemma nel capitolo 4.6.2 si osserva che affinché la maggiorazione valga è necessario 
che $r<\min\{\min_{j} \{r_{B_j}\}, r_c \}$.
Consideriamo ora la funzione $$\nu=\frac{r-s-\sqrt{(r-s)^2-2tCrmn}}{mn}$$ e ricordiamone alcune proprietà:
\begin{enumerate}[1.]

\item
E' interessante perché essa alla conclusione della dimostrazione del teorema di CK 
permette di scrivere in forma compatta la soluzione del problema maggiorante nella seguente forma: 
$$u = \nu(x_1+\ldots+x_{n-1}, t)[1,\ldots,1]^T$$

\item
Essa è analitica in un intorno dell'origine, in particolare per $t<\frac{(r-s)^2}{2Crmn}$ e di 
conseguenza anche in $B_h(0,0)$ con $h=\frac{r}{8Cmn}$.

\item
In $B_h(0,0)$ vale la condizione $s^2+m\nu ^2 (s,t)< r^2$

\item
Unendo le ultime due condizioni si ottiene che la soluzione è maggiorante in 

\end{enumerate}

\section{Versione per EDP non lineari}
Versione per EDP non lineari
Riscrivere l'equazione come un problema di evoluzione (vedi pdf)


