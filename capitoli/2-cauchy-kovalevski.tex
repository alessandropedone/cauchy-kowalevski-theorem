\chapter{Teorema di Cauchy-Kovalevski}

Perché non Kovalevskaya
Chi è K
Cronologia delle dimostrazioni
Abbreviato CK
\section{Versione per EDO}
Versione per EDO
\section{Versione per EDP quasi-lineari}
Versione per EDP quasi-lineari
Rifacendoci a Evans, e quindi anche usando la notazione in esso presente, 
assumiamo che i coefficienti del sistema $B_j$ e $c$ abbiano come raggi di convergenza $r_{B_j}>0$ e $r_c>0$ 
di conseguenza per il Lemma nel capitolo 4.6.2 si osserva che affinché la maggiorazione valga è necessario 
che $r<\min\{\min_{j} \{r_{B_j}\}, r_c \}$.
Consideriamo ora la funzione $$\nu=\frac{r-s-\sqrt{(r-s)^2-2tCrmn}}{mn}$$ e ricordiamone alcune proprietà:
\begin{enumerate}[1.]

\item
E' interessante perché essa alla conclusione della dimostrazione del teorema di CK 
permette di scrivere in forma compatta la soluzione del problema maggiorante nella seguente forma: 
$$u = \nu(x_1+\ldots+x_{n-1}, t)[1,\ldots,1]^T$$

\item
Essa è analitica in un intorno dell'origine, in particolare per $t<\frac{(r-s)^2}{2Crmn}$ e di 
conseguenza anche in $B_h(0,0)$ con $h=\frac{r}{8Cmn}$.

\item
In $B_h(0,0)$ vale la condizione $s^2+m\nu ^2 (s,t)< r^2$q

\item
Unendo le ultime due condizioni si ottiene che la soluzione è maggiorante in 

\end{enumerate}

\section{Versione per EDP non lineari}
Versione per EDP non lineari
Riscrivere l'equazione come un problema di evoluzione (vedi pdf)
\section{Versione astratta}
Versione astratta

\section{Esempi}
Dopo aver visto il teorema di Cauchy-Kovalevski in tutte le sue forme più note, si concentra ora lo sguardo su tre esempi importanti  che aiutano a inquadrare meglio il ruolo che giocano le ipotesi e che limiti ha questo teorema.

Tale discussione risulta particolarmente di rilievo, poiché per molto tempo si ritenne ragionevole pensare che un'equazione differenziale con coefficienti piuttosto regolari, come ad esempio $C^\infty$, dovesse avere almeno una soluzione. Questo, però, oltre al caso di analiticità trattato dal teorema oggetto del capitolo, in generale non accade.

\setcounter{example}{0}

\begin{example}[\textbf{Lewy}]
Questo primo esempio è decisamente il più importante ed interessante tra quelli qui trattati, 
proprio perché permette di introdurre in modo più rigoroso il problema appena citato.

Nel 1957 Hans Lewy propose questo semplice controesempio, volto a mostrare come l'ipotesi di \textbf{analiticità} nel teorema di 
Cauchy-Kovalevki fosse cruciale, portando un caso di un operatore differenziale lineare con coefficienti analitici che necessita 
della presenza di una forzante anch'essa analitica per possedere delle soluzioni almeno $C^1$.

Ciò mostra come sia cruciale, non solo una discussione sulle condizioni sufficienti per l'esistenza di soluzioni locali, 
ma anche una sulle condizioni necessarie. Infatti, come si vedrà nella sezione \ref{Hormander}, Hormander, 
matematico che contribuì ampiamente alla teoria delle equazioni lineari, 
rispose all'emersione di questo problema proprio con delle condizioni necessarie per l'esistenza di soluzioni locali 
(e quindi anche globali!) per equazioni lineari.

Preliminarmente si riportano qui sotto gli enunciati di due teoremi che torneranno utili nella discussione:

\begin{namedtheorem}[Formula di Green in $\mathbb{C}$]
\hpth{
D \subseteq \mathbb{C} \text{ dominio regolare }\\
f:D \rightarrow \mathbb{C}\\
f \in H(\interior{D})
}
{\oint\limits_{\partial^+D}f(z)\,dz=2i\iint\limits_D\frac{\partial f}{\partial \overline{z}}(x+iy)\,dxdy}
\end{namedtheorem}

\begin{namedtheorem}[Principio di riflessione di Schwarz]
\hpth{
D \subseteq \mathbb{C} \text{ simmetrico rispetto a } \mathbb{R}\\
f:D \rightarrow \mathbb{C}\\
f(\mathbb{R} \cap D) \subseteq \mathbb{R}\\
f \in H(\interior{D})
}
{f(\overline{z})=\overline{f(z)} \quad \forall z \in \interior{D}}
\end{namedtheorem}

Per entrare nel vivo dell'esempio, si definisce il seguente operatore:
$$L=D_x+iD_y-2i(x+iy)D_t$$
che soddisfa le proprietà precedentemente enunciate e il cui comportamento peculiare emerge dal teorema che si enuncia di seguito.

\begin{theorem}
\hpth{
f \text{ funzione continua a valori reali che dipende solo da } \; t\\
u\in C^1\;:\;Lu=f \text{ in un intorno dell'origine }
}
{f \text{ analitica in un intorno di } t=0}
\end{theorem}

\begin{proof}
Innanzitutto si fissa un $R>0$ tale che $\{(x,y,t): x^2+y^2<R^2,|t|<R\}$ sia contenuto nell'intorno dell'origine delle ipotesi (ovviamente questo $R$ esiste sempre) e si procede seguendo questi passi:
\begin{enumerate}[1.]
\item
Si definisce la funzione: 
\begin{equation*}
V(t,s)=\int\limits_{\gamma_r}u(x,y,t) \, dz \quad \text{con} \quad
\begin{system}
t \in (-R,R)\\
r^2=s \in [0,R^2)\\
\gamma_r=\partial^+B_r(0,0)\\
z=x+iy
\end{system}
\end{equation*}
\item
Si ricerca una relazione tra $V_s$ e $V_t$:
\begin{align*}
V&=i\iint\limits_{B_r(0,0)}(u_x+iu_y)(x,y,t) \, dx \, dy &\text{per formula di Green}\\
&=i\int_0^r \int_0^{2\pi} (u_x+iu_y)(\rho \cos \theta,\, \rho \sin \theta,\, t) \, \rho \,d\rho \, d\theta &\text{in coordinate polari}\\
V_r&=i\int_0^{2\pi} (u_x+iu_y)(\rho \cos \theta,\, \rho \sin \theta,\, t) \, r \, d\theta &\text{derivando}\\
&=\int\limits_{\gamma_r}(u_x+iu_y)(x,y,t) \, r \, \frac{dz}{z}\\
V_s&=\frac{1}{2r}V_r=\int\limits_{\gamma_r}(u_x+iu_y)(x,y,t) \, \frac{dz}{2z}\\
&=\int\limits_{\gamma_r}u_t(x,y,t) \, dz + \int\limits_{\gamma_r}f(t) \, \frac{dz}{2z} &\text{usando } Lu=f\\
&=iV_t + \pi i f(t) \numberthis \label{eq:4}
\end{align*}
\item
Si definiscono le funzioni:
\begin{align*}
F(t)&=\int_{0}^{t} f(\tau) \, d\tau\\
U(t,s)&=V(t,s)+\pi F(t)\;.
\end{align*}
e si osservano le seguenti proprietà di $U$ vista come funzione di $w=t+is$: 
\begin{itemize}
\item
si verifica che soddisfa l'equazione di Cauchy-Riemann $U_t+iU_s=2U_{\overline{z}}=0$ utilizzando la relazione (\ref{eq:4}),
\item
olomorfa per $(s,t) \in (0,R^2) \times (-R,R)$ per la proprietà precedente,
\item
continua per $(s,t) \in [0,R^2) \times (-R,R)$ perché lo è $V$,
\item
$U(0,t)=\pi F(t)$ per $t\in (-R,R)$, ovvero assume valori reali sull'asse reale.
\end{itemize}
\item
Si prolunga analiticamente $U$ in un intorno dell'origine, infatti, 
date le proprietà appena osservate, valgono le ipotesi del principio di riflessione di Schwarz che ci permette 
di definire U per $s\in (-R^2,0)$ con la seguente formula: $$U(t,s)=\overline{U(t,-s)}.$$
\item
Si conclude il ragionamento notando che, se il prolungamento di $U$ è analitico in un intorno dell'origine, lo deve essere anche $U(t,0)=\pi F(t)$ e anche $f=F'$.
\end{enumerate}
\end{proof}

\textbf{Generalizzazione.} Il teorema appena trattato si presta in realtà anche a una generalizzazione interessante e l'idea è la seguente: si cerca di mostrare che, nonostante la forma caratteristica di $L$ non abbia punti singolari, è possibile scegliere una forzante $F \in C^{\infty} (\mathbb{R}^3, \mathbb{R})$ in modo tale che \textbf{ovunque} l'equazione differenziale $Lu=F$ non ammetta soluzioni.

\begin{remark}
con la notazione $C^{k} (\mathbb{R}^n, \mathbb{R}^m)$ con $k \in \mathbb{N} \cup \{\infty\}$ si indica l'insieme delle funzioni $C^{k}$ del tipo $h:\mathbb{R}^n \rightarrow \mathbb{R}^m$.
\end{remark}

Prima di scendere nello specifico
\begin{itemize}
\item
un insieme $E$ è senza parte interna se $\interior{E}=\emptyset$
\item
teorema della categoria di Baire per gli spazi metrici ci dice che gli spazi metrici completi sono tutti Spazi di Baire (in senso topologico)
\end{itemize}
si richiama l'enunciato
\begin{namedtheorem}[Teorema della categoria di Baire]
\hpthth{
(X,d) \text{ spazio metrico completo }\\
\{A_n\}_{n \in \mathbb{N}} \subseteq 2^X \text{ famiglia di insiemi aperti densi in } X\\
\{E_n\}_{n \in \mathbb{N}} \subseteq 2^X \text{ famiglia di insiemi chiusi senza parte interna }
}
{
\bigcap\limits_{n \in \mathbb{N}} A_n \text{ è denso in X }
}
{
\bigcup\limits_{n \in \mathbb{N}} E_n \text{ è senza parte interna }
}
\end{namedtheorem}

In particolare si è interessati a un'applicazione della contronominale della seconda tesi:

\begin{namedtheorem}[Corollario (argomento per assurdo di Baire)]
\hpth{
(X,d) \text{ spazio metrico completo }\\
\{E_n\}_{n \in \mathbb{N}} \subseteq 2^X \text{ famiglia di insiemi chiusi }\\
X=\bigcup\limits_{n \in \mathbb{N}} E_n
}
{
\exists \, n \in N \text{ tale che } \interior{E_n} \neq \emptyset
}
\end{namedtheorem}
Questo teorema infatti ha un'applicazione: dimostrare che uno spazio metrico completo è unione numerabile di insiemi mai densi porta a contraddizione

Si divide tutto in 4 passi:
\begin{enumerate}
\item
traslare il problema del teorema precedente in modo da ricondursi al caso di un generico punto $(x_0,y_0,t_0)$, usando come forzante la funzione $g(x,y,t)=f(t+2xy_0-2x_0y)$ e sfruttando l'invarianza dell'operatore $L$ rispetto a $T(x,y,t)=(x-x_0,y-y_0,t-t_0-2xy_0+2x_0y)$, ovvero la validità della relazione: $L(u \,\circ\, T)=(Lu) \,\circ\, T$.
\item
costruire la funzione $F_\epsilon$
\item
lemma tecnico su $E_{j,n}$
\item
svolgere la dimostrazione sfruttando i punti precedenti
\end{enumerate}

Servono i seguenti lemmi

\begin{theorem}
\hpth{
f \in C^{\infty} (\mathbb{R},\mathbb{R})\\
(x_0,y_0,t_0)\in \mathbb{R}^3\\
u\in C^1\;:\;Lu(x,y,t)=f(t+2xy_0-2x_0y) \text{ in un intorno di } (x_0,y_0,t_0)\\
}
{f \text{ analitica in un intorno di } t=t_0}
\end{theorem}

\begin{theorem}
\hpthth{
\{(x_i,y_i,t_i)\}_{i \in \mathbb{N}} \text{ denso in } \mathbb{R}^3\\
c_i=2^{-i}e^{-\rho_i} \text{ con } \rho_i=|x_i|+|y_i| \quad \forall i \in \mathbb{N}\\
a=\{a_n\}_{n \in \mathbb{N}} \in l^{\infty}\\
f \in C^{\infty} (\mathbb{R},\mathbb{R}) \text{ periodica e non analitica }
}
{F_a(x,y,t)=\sum_{i \in \mathbb{N}} a_ic_if'(t+2xy_0-2x_0y) \text{ converge uniformemente in } \mathbb{R}^3}
{\text{lo stesso vale per le derivate formali } D^{\alpha}F_a=\sum_{i \in \mathbb{N}} a_ic_iD^{\alpha}f'(t+2xy_0-2x_0y)}
\end{theorem}

\begin{remark}
Prima di proseguire è utile soffermarsi brevemente su due questioni:
\begin{itemize}
\item
$l^{\infty}$ è una spazio di Banach se dotato della norma: $||b||_\infty=\sup_i|b_i|$ per ogni $b \in l^{\infty}$;
\item
$f$ con le proprietà delle ipotesi esiste, per esempio la funzione $$f(x)=\sum_{n \in \mathbb{N}}\frac{\cos(n!\,x)}{(n!)^n}$$ è definita da un serie puntualmente convergente e $C^{\infty}(\mathbb{R},\mathbb{R})$, inoltre è periodica di periodo $2\pi$ e si può dimostrare che essa non è analitica in nessun punto $x\in\mathbb{R}$. Sopratutto per quest'ultimo aspetto si veda \cite{John} per maggiori dettagli.
\end{itemize}
\end{remark}



\end{example}



\begin{example}[\textbf{Kovalevski}]

Questo esempio, dovuto a Kovalevski stessa, è utile a comprendere più a fondo, in modo quanto più essenziale possibile, l'importanza, o meglio la necessità, di assumere che la superficie scelta per assegnare i dati di Cauchy sia \textbf{non-caratteristica} per l'equazione differenziale in osservazione.
 
Si consideri quindi il seguente problema di Cauchy per l'equazione del calore in una dimensione:
\begin{align} 
\label{eq:1}
u_t-u_{xx}&=0\\ 
\label{eq:2}
u(x,0)&=\frac{1}{1+x^2} \quad \forall \, x \in \mathbb{R}
\end{align}
\begin{remark}
La condizione per $u_x$ su $\Gamma$ necessaria per completare il problema di Cauchy è già implicitamente imposta dall'equazione
(\ref{eq:2}).
\end{remark}
L'obiettivo che ci si pone è quello di dimostrare che non ammette soluzioni analitiche in un intorno dell'origine.

\begin{enumerate}
\item
Per cominciare si osserva che in questo caso la superficie su cui sono stati assegnati i dati di Cauchy $(1.2)$ è 
$\Gamma=\left\lbrace(x,t) \in \mathbb{R}^2:t=0\right\rbrace$. Essa in ogni punto ha come versore normale $(0,1)$ ed è, quindi,
caratteristica per l'equazione (\ref{eq:1}), poiché
$$\sum_{|\alpha|=2}^{\;} a_\alpha \boldsymbol{\nu}^\alpha = a_{(2,0)}\boldsymbol{\nu}^{(2,0)} = 0.$$
\item
Per assurdo si supponga di avere una soluzione del problema $u$ analitica in un intorno dell'origine, ovvero:
$$u(x,t) = \sum_{\alpha = (\alpha_1, \alpha_2) }^{\;} c(\alpha) \, x^{\alpha_1} \, t ^{\alpha_2}, \quad
 c(\alpha) = \frac{D^\alpha u(0,0)}{\alpha!}$$
dove $|(x,t)|<r$ per qualche $r>0$.
\item
Si calcolano i valori dei coefficienti $c(2n,0)$ $\forall n \in \mathbb{N}$.\\
Per fare questo si sviluppa in serie di potenze centrata nell'origine la funzione del problema di Cauchy:
$$\frac{1}{1+x^2} = \frac{d}{dx}\arctan(x) = \frac{d}{dx}\sum_{n=0}^{\infty}\frac{(-1)^n}{2n+1} \, x^{2n+1} 
= \sum_{n=0}^{\infty}(-1)^n \, x^{2n} \quad \forall x \in \mathbb{R}.$$
Da questa serie si ottengono le seguenti relazioni:
\begin{align*}
D_x^{2n}u(0,0) &= \frac{d^{2n}}{dx^{2n}} \left. \frac{1}{1+x^2}\right|_{x=0} = (-1)^n (2n)!\\
D_x^{2n+1}u(0,0) &= \frac{d^{2n+1}}{dx^{2n+1}} \left. \frac{1}{1+x^2} \right|_{x=0} = 0
\end{align*} 
dalle quali si ricava: $c(2n,0)=(-1)^n$ e $c(2n+1,0)=0$.
\item
Si calcolano i valori dei coefficienti $c(2n,n)$ e si dimostra che  $c(2n,n) \xrightarrow{n\rightarrow\infty} +\infty$.\\
A questo scopo, invece, si sfrutta l'equazione $(1.1)$ per ottenere la seguente relazione tra i coefficienti:
\begin{equation} 
\label{eq:3}
c(\alpha_1,\alpha_2+1) = \frac{(\alpha_1+2)(\alpha_1+1)}{(\alpha_2+1)} \, c(\alpha_1+2,\alpha_2).
\end{equation}
E si utilizza, quindi, quest'ultima come segue:
\begin{align*}
c(2n,n) &= \frac{(2n+2)(2n+1)}{n} \, c(2n+2,n-1)   &(\ref{eq:3})\quad\text{con} \quad 
\begin{system}
\alpha_1=2n\\
\alpha_2+1=n
\end{system}\\
 &= \ldots = \frac{(2n+2n)\cdots(2n+1)}{n!} \, c(2n+2n,0) &\text{\quad iterando su } n\\
 &= \frac{(4n)!}{(2n)! \, n!} (-1)^{2n} \\
 &\sim \frac{1}{\sqrt{\pi n}}\left(\frac{64n}{e}\right)^n \xrightarrow{n\rightarrow\infty} +\infty  &\text{per la formula di Stirling}
\end{align*}
\item
Si completa il ragionamento in modo immediato osservando che 
$c(2n,n) \, x^{2n} \, t ^{n}\xrightarrow{n\rightarrow\infty} +\infty$ $\forall (x,t) \neq (0,0)$, 
infatti ciò implica direttamente che la serie di potenze non converge in ogni punto diverso dall'origine e questo è assurdo.
\end{enumerate}
\end{example}


\begin{example}[\textbf{Hadamard}]
L'esempio che si propone ora, dovuto ad Hadamard, aiuta a capire un limite importante del teorema di Cauchy-Kovalevski, ovvero il fatto che esso non fornisca alcun controllo sulla \textbf{relazione} tra i dati di Cauchy e la forma della soluzione analitica,la quale potrebbe risultare instabile.

Per osservare ciò si considera il seguente problema di Cauchy per l'equazione di Laplace in due dimensioni al variare di $n$:
\begin{align*}
u_{xx}+u_{yy}&=0\\
u(x,0)&=0 \numberthis \label{eq:6}\\ 
u_y(x,0)&=n\sin(nx)e^{-\sqrt{n}} \quad \text{con} \quad n\in\mathbb{N}
\end{align*}


L'obiettivo che ci si pone in questo caso è quello di mostrare come al crescere di $n$ 
si verifica un blow-up della soluzione $u_n$ del problema di Cauchy (\ref{eq:6}).
\begin{enumerate}[1.]
\item
Il problema, come nell'esempio precedente, è assegnato su $\Gamma=\left\lbrace(x,y) \in \mathbb{R}^2:y=0\right\rbrace$, che è naturalmente una superficie non caratteristica per l'equazione di Laplace (si noti infatti che essa è ellittica).
\item
E' facile verificare che la funzione $u_n(x,y)=\sin(nx)\sinh(ny)e^{-\sqrt{n}}$ soddisfa il problema di Cauchy e che essa è analitica, per questo essa anche l'unica possibile con quest'ultima proprietà.
\item
Si osserva, infine, come $\sinh(ny)e^{-\sqrt{n}}\xrightarrow{n\rightarrow\infty} \infty$.
\end{enumerate}
Come conclusione di questa discussione è interessante osservare anche come la soluzione non dipenda con continuità dai dati. 
Infatti, considerando il problema di Cauchy per $n=\infty$, ovvero con dati $u(x,0)=u_y(x,0)=0$, si nota immediatamente che l'unica soluzione analitica è $u\equiv0$, la quale è profondamente diversa dal comportamento asintotico di $u_n$.

\end{example}

