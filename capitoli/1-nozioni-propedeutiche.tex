\chapter{Nozioni propedeutiche}
In questo capitolo vengono trattate delle nozioni fondamentali per la trattazione della teoria locale presentata.\\
Si tratta il metodo delle caratteristiche che sarà cruciale nella dimostrazione del teorema di CK anche se in una forma meno generale di quella presentata

\section{Equazioni differenziali alle derivate parziali}
\section{Superfici caratteristiche}
Cosa si intende per superficie analitica\\
Definizione di superficie caratteristica

\begin{enumerate}[i.]
\item
caso $t=0$ e calcolo di tutte le derivate (evans)
\item
caso lineare (folland)
\item
caso quasi lineare (folland)
\item
caso generale e calcolo di tutte le derivate (evans)
\item 
classificazione delle EDP
\end{enumerate}


\section{Metodo delle caratteristiche}

\begin{enumerate}[i.]
\item
caso lineare (folland)
\item
caso quasi lineare (folland)
\item
caso generale (evans)
\end{enumerate}