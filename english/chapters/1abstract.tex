\chapter*{Abstract}
\addcontentsline{toc}{chapter}{Abstract}

In 1874, Sofya Kowalevski, the first woman to obtain a doctorate in mathematics in Europe, brought to light the proof of the Cauchy-Kowalevski theorem (CKT), the first general result for the existence of local analytic solutions to partial differential equations (PDEs) with Cauchy data.

The thesis aims to present this milestone of mathematics, highlighting the depth of detail, consequences, and the simplicity of the ideas it brought to light. To this end, recurring references to fundamental notions and results are made to address the topic, and all the main forms in which the CKT can be stated are also discussed.

Additionally, there is a section dedicated to three historically crucial examples for understanding PDEs and another dedicated to two fundamental applications of the CKT: the Holmgren theorem and the Cartan-Kähler theorem.

\vspace{6mm}
\textbf{Keywords:} PDE, characteristics, analyticity/holomorphy, power series, majorant method, Cauchy-Kowalevski, Holmgren, and Cartan-Kähler theorems

\newpage
\blankpage
