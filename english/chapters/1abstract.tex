\chapter*{Abstract}
\addcontentsline{toc}{chapter}{Abstract}

In 1874, Sofya Kowalevski, the first woman to obtain a doctorate in mathematics in Europe, brought to light the proof of the Cauchy-Kowalevski theorem (CKT), the first general result for the existence of local analytic solutions to partial differential equations (PDEs) with Cauchy data.

\vspace{6mm}
The thesis aims to present this milestone of mathematics by highlighting its detailed depth, consequences, and also the simplicity of the ideas it brought to light. To this end, there are recurring references to fundamental notions and results to address the topic, and moreover, all the main forms in which the TCK can be stated are treated.

\vspace{6mm}
Additionally, there is a section dedicated to three historically crucial examples for understanding PDEs and another dedicated to its two fundamental applications: the Holmgren theorem and the Cartan-Kähler theorem.

\vspace{6mm}
\emergencystretch 3em
\textbf{Keywords:} PDEs, characteristics, analyticity/holomorphy, power series, majorants method, Cauchy-Kowalevski, Holmgren, and Cartan-Kähler theorems

\newpage
\blankpage
