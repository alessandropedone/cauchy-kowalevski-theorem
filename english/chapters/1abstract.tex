\chapter*{Abstract}
\addcontentsline{toc}{chapter}{Abstract}

Sofya Kowalevski, the first woman to earn a PhD in mathematics in Europe, in 1874 provided the proof of the Cauchy-Kowalevski theorem (CKT), the first general result for the existence of local analytic solutions to partial differential equations (PDE) with Cauchy data.

\vspace{6mm}
This thesis aims to present this milestone in mathematics, highlighting the depth of detail, the consequences, and also the simplicity of the ideas that allowed it to emerge. To this end, recurring references to fundamental notions and results are made to address the topic, and all the main forms in which the CKT can be stated are also discussed.

\vspace{6mm}
In addition, there is a section dedicated to three historically crucial examples for the understanding of PDEs, and another dedicated to two fundamental applications of the CKT: the Holmgren theorem and the Cartan-Kähler theorem.

\vspace{6mm}
\textbf{Keywords:} PDE, characteristics, analyticity/holomorphy, power series, majorant method, Cauchy-Kowalevski, Holmgren, and Cartan-Kähler theorems

\newpage
\blankpage
