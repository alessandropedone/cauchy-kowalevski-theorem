\chapter{Introduzione}

\section{Chi era Kowalevski?}

Sofya Vasilyevna Kovalevskaya (1850-1891) è stata una matematica russa. Per varie ragioni, tra cui proprio il teorema al centro di questa discussione, è tutt’ora una delle figure femmili più rilevanti per la storia di questa disciplina.

Prima di tutto, è importante sottolineare che qui in poi ci riferirono a lei con il nome con coi soleva firmarsi nelle sue pubblicazioni, ovvero Kowalevski.

Per allontanarsi dalla Russia dovette prendere parte a un matrimonio bianco, difatti sposò un uomo con cui poi non ebbe alcun reale rapporto sentimentale e da cui rimase spesso anche geograficamente distante.

Tutto ciò le permise di continuare a studiare in Germania; ed è qui che conobbe Karl Weierstrass, uno dei matematici più influenti del suo tempo.

Dopo un prima conoscenza avvenuta nello studio del professore, il rapporto tra loro continuò a svilupparsi grazie alle evidenti doti matematiche di Sofya, che Weierstrass non poté fare a meno di assecondare. Infatti continuò a impartirle lezioni private, fino ad arrivare a supervisionare il suo lavoro di ricerca.

Per quanto riguarda le idee politiche di Kowalevski, possiamo affermare con certezza storica la sua vicinanza a movimenti femministi e idee socialiste e radicali, le quali possono essere fatte risalire al suo background familiare e alle spunti che colse durante la sua esperienza di vita negli stati dell'odierna Europa. È certamente degno di nota il fatto che ricevette regalo da Anna, sua sorella, diverse copie di riviste radicali di quel tempo, che discuteva del cosiddetto nichilismo antico
\footnote{Per i nichilisti antichi la scienza, e non la religione e la superstizione, appariva come il mezzo più efficace per aiutare la popolazione a condurre e rappresentava, quindi, verità e progresso.}.

Quello su cui vogliamo concentrare la nostra attenzione non sono le sue idee politiche, sociali e filosofiche, ma piuttosto sul suo contributo alla matematica. Kowaleski, con l'aiuto di quello che possiamo chiamare il suo mentore, arrivò a diverse scoperte importanti. Dopo diversi anni di collaborazione, arrivò a pubblicare ben tre tesi di dottorato in un solo anno: il 1874. Ma questo non è l'unico aspetto notevole, infatti lei fu la prima donna a conseguire un dottorato; ciò fu possibile grazie al supporto di Weierstrass, come emerge da una lettera che scrisse a Fuchs, un suo collega all'Università di Berlino, a riguardo dell'approvazione delle tesi di Sofya. Inoltre le sue pubblicazioni, oltre che significative per quanto appena detto, si rivelarono delle pietre miliari della matematica, le quali 
In particolari i temi trattati sono:
\begin{itemize}
\item Equazioni differenziali alle derivate parziali (EDP), teorema di Cauchy-Kowalevski
\item Meccanica, Kowalevski top
\item Integrali ellittici
\end{itemize}


Dopo il successo, coronato anche da alcuni premi, che naturalmente seguì la pubblicazione di queste ricerche ritornò per un periodo in Russia, scelta che però si rivelerà di fatto inutile per il proseguimento della sua carriera accademica. Difatti, quando il marito a cui doveva la possibilità di aver studiato in Germania venne a mancare, si trasferì in Svezia, dove collezionò un altro primato: divenne la prima donna al mondo professoressa di matematica, ottenendo la cattedra all'Università di Stoccolma.
Purtroppo la sua vita venne interrotta prematuramente all'età di 41 da una polmonite, la quale, considerando ciò che emerge dalle fonti, le impedì di portare avanti una sua grande passione: la produzione letteraria.

Nonostante non abbia potuto esprimersi come avrebbe voluto in quest'ambito, esistono numerose sue rappresentazioni artistiche, sia in letteratura che nel cinema.

Citiamo di seguito le opere cinematografiche principali:
\begin{itemize}
\item `` Sofya Kowalevski'' (1985, 3 episodi,
218 minuti, biografico, Lenfilm, OTF, colore). Gran Premio al Festival
Internazionale del Film Televisivo Multiepisodio di Pianciano Terme,
Italia, nel 1985. Ayan Gasanovna Shakhmaliyeva (1932-1999) regista
originaria dell'Azerbaijan.
\item A Hill on the Dark Side of the Moon (Swedish: Berget på månens baksida)
is a 1983 Swedish drama film directed by Lennart Hjulström
\end{itemize}

Citiamo di seguito le opere letterarie principali:
\begin{itemize}
\item L'anno dopo la sua morte, la sua cara amica Anne Charlotte Leffler,
sorella del matematico Gösta Mittag-Leffler e moglie dell'algebrista
italiano Pasquale del Pezzo, le dedicò una biografia (Sonja Kovalevsky.
Ciò che ho vissuto con lei e ciò che mi ha detto di sé, Ed. Albert
Bonniers, Stoccolma, 1892)
\item Little Sparrow: A Portrait of Sophia Kovalevsky (1983), Don H. Kennedy,
Ohio University Press, Athens, Ohio 
\item Beyond the Limit: The Dream of Sofya Kovalevskaya (2002) a biographical
novel by mathematician and educator Joan Spicci, published by Tom
Doherty Associates, LLC
\item `` Too Much Happiness'' (2009), short story
by Alice Munro, published in the August 2009 issue of Harper's Magazine
(ispirato dal primo, infatti il racconto ripercorre gli ultimi giorni
di vita di Sof'ja Kovalevskaja arricchito da reminiscenze del passato
che Munro ha acquisito da lettere, diari e scritti. Munro ha potuto
accedere a tali documenti tramite la moglie di Don H. Kennedy la quale
è una lontana discendente di Kovalevskaja)
\end{itemize}

\section{Il teorema di Cauchy-Kowalevski}

Una volta introdotta la figura storia, possiamo ora fare il primo passo verso la scoperta di una delle ricerche di Sofya: il teorema di Cauchy-Kowalevski, che da qui in poi capiterà di abbreviare con l'acronimo TCK.

Innanzitutto descriviamo rapidamente il contesto scientifico, per quanto riguarda l'ambito delle EDP, di quel tempo. 

Padre della ricerca di nostro interesse che si svolgeva nell'Ottocento è Augustin-Louis Cauchy, un matematico che sarà sicuramente noto al lettore. In quegli anni, in particolare tra il 1835 e il 1842, Cauchy si stava occupando di sviluppare la teoria delle funzioni olomorfe, già avviata da altre grandi personalità di spicco come Eulero, Laplace e Fourier.

Cauchy ebbe l'intuizione di applicare questi risultati alle equazioni differenziali. 

Quello che è importante cogliere, cercando di calarsi nella mentalità di quel periodo, è che la teoria classica e le serie di potenze erano strumenti molto promettenti, in primis per la loro semplicità ed eleganza, ma anche per la potenzialità di approssimazione che racchiudeva un semplice tronchetto di una serie.

Il tenativo di Cauchy di applicare alle equazioni differenziali gli strumenti ottenuti dalle sue ricerche fu un successo, ma soltanto parziale per una semplice ragione: Cauchy non riuscì ad andare oltre lo studio di equazioni differenziali ordinarie (EDO) e di EDP lineari. 

Il salto avvenne proprio grazie a Kowalevski e Weierstrass. Quest'ultimo è molto ottimista sui risultati che pensava si potessero raggiungere, forse ancora più di Cauchy: basti pensare che enunciò una congettura secondo la quale sarebbe stato possibile definire funzioni analitiche tramite equazioni differenziali, grazie a serie potenze formali ricavate dall'espressione dell'equazione.

Per questo spinse Kowalevski, insieme al suo talento, verso questo tema, nel quale lei seppe indagare molto più a fondo.

È sbagliato, però, pensare che le guida di Sofya furono soltanto Cauchy e Weierstrass: altri matematici si dedicarono a queste tematiche, tra i quali ricordiamo tra i più importanti Briot, Bouquet e Fuchs che svilupparono meglio i concetti di singolarità e Jacobi che fornì per primo la definizione di equazione in forma normale
\footnote{quest'ultimo in particolare si rivelerà un elemento cruciale nella ricerca di Kowalevski}.

Da queste basi, l’idea importante avuta da Kowalevski può essere riassunta in questo modo: 
\begin{enumerate}
\item attuare un cambio di variabile che permettesse di scrivere un'equazione non lineare in forma normale (si veda il testo per il significato di questo termine), mantengono le ipotesi di regolarità sui dati, e di potersi occupare dell’esistenza di una soluzione a questo sistema;
\item trasformare un'equazione qualsiasi in forma normale in un sistema quasi-lineare particolari;
\item applicare il metodo dei maggioranti già utilizzato da Cauchy per le sue scoperte su EDO ed EDP lineari.
\end{enumerate}
Come accade spesso in matematica, la dimostrazione venne poi semplificata da E. Goursat in un suo libro di testo di analisi matematica risalente intorno al 1900. Inoltre nel corso del tempo vennero proposti enunciati e dimostrazioni più astratti e più generali, grazie al lavoro di Ovsyannikov, Treves e Nierenberg.

Notiamo rapidamente anche che Darboux raggiunse risultati molto simili a Kowalevski, ma con meno generalità, nello stesso periodo, infatti entrambi pubblicarono le loro ricerche nel 1874.

Alla luce di quanto detto fino ad ora, ci poniamo alcune domande cruciali, a cui vogliamo trovare risposte quanto più esaustive possibile e che svolgeranno il ruolo di guida per il discorso che vogliamo affrontare:
\begin{itemize}
\item è possibile che esista una soluzione analitica a un sistema di EDP con dati di Cauchy?
\end{itemize}
se sì
\begin{itemize}
\item sotto quali ipotesi?
\item è unica?
\item dipende in modo continuo dai dati iniziali?
\item quali applicazioni ha questo teorema?
\end{itemize}



