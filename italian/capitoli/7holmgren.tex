\chapter{Conseguenze}
\section{Teorema di Holmgren}

\section{Altre applicazioni}
teorema inutile nella pratica che però ha ispirato ricerche e scoperte future cruciali e utili (come accadde per i grafi di eulero)

I. Ekeland

Cartan Kahler

4 (with Chiappori), "Aggregation and market demand: an exterior differential calculus viewpoint", Econometrica, 67 (1999), p. 1435-1458

This paper solves a basic problem in economic theory, which had remained open for thirty years, namely the characterization of market demand functions. The method of proof consists of reducing the problem to a system of nonlinear PDEs, for which convex solutions are sought. This is rewritten as an exterior differential system, and is solved by the Cartan-Kähler theorem, together with some algebraic manipulations to achieve convexity. The introduction of exterior differential calculus proved to be a breakthrough, and was the starting point of a long collaboration with P.A. Chiappori. We realized that the mathematical structure we had discovered in this problem was to be found also in one of the major problems of econometrics: given a group (a household, for instance), can one characterize and identify the preferences of each member if one observes only the collective demand ?. I am happy to say that this research program is now concluded, with the publication of two major papers [14] and [25] and a100-pages survey [28] which will probably turn into a book.

14: (with P.A. Chiappori) "The microeconomics of group behaviour: general characterization". Journal of Economic Theory, september 2006 , volume 130 (p.1 – 26)

25: (with P.A. Chiappori) "The microeconomics of group behaviour: Identification". Econometrica, 2005, 44 pages

28: (with P.A. Chiappori). "The mathematics and economics of aggregation". Foundations and Trends in Economic Theory, 2009