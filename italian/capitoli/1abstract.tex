\chapter*{Abstract}
\addcontentsline{toc}{chapter}{Abstract}

Sofya Kowlevski, la prima donna ad conseguire un dottorato in matematica in Europa, nel 1874 dava la luce alla dimostrazione del teorema di Cauchy-Kowalevski (TCK), il primo risultato generale per l'esistenza di soluzioni locali analitiche per equazioni differenziali alle derivate parziali (EDP) con dati di Cauchy.

\vspace{6mm}
La tesi mira a presentare questa pietra miliare della matematica esaltandone la profondità del dettaglio, le conseguenze e anche la semplicità delle idee che ha permesso di far emergere. A questo scopo sono ricorrenti i richiami di nozioni e risultati fondamentali ad affrontare il discorso e vengono trattate tutte le forme principali in cui è possibile enunciare il TCK.

\vspace{6mm}
A completamento sono presenti anche una sezione dedicata a tre esempi storicamente cruciali alla comprensione delle EDP e un'altra dedicata, invece, alle due sue fondamentali applicazioni: il teorema di Holmgren e il teorema di Cartan-Kähler.

\vspace{6mm}
\textbf{Parole chiave:} EDP, caratteristiche, analiticità/olomorfia, metodo dei maggioranti, teoremi di Cauchy-Kowalevski, Holmgren e Cartan-Kähler

\newpage
\blankpage