\chapter{Sviluppi successivi}

Ci sono varie versioni di questo teorema e

Le conseguenze di questo teorema si osservano in vari campi, tra cui i principali sono:
\begin{itemize}
\item teoria delle equazioni differenziali
\item fisica matematica: emersione di numerose domande (cosa succede nella realtà se esiste una sol. analitica locale?)
\item geometria differenziale
\item teoria economica
\end{itemize}

Impatto sulla teoria delle equazioni differenziali:
\begin{itemize}
\item confutare la congettura di Weierstrass
\item teorema di Holmgren
\item ricerca di condizioni necessarie e/o sufficienti per l'esistenza di soluzioni locali di Treves e Nirenberg
\item teoria degli operatori differenziali lineari di Hörmander
\end{itemize}
\section{Versioni alternative}

Ispirazione a Ovsyannikov, L.V. (vedi bibliografia) (vedi pag 109 e 110 treves analytic)


Versioni alternative

\begin{center}
\normalsize Versione astratta \\
\footnotesize\textit{(classi di Ovsyannikov)}\\
\normalsize $$\big\Downarrow$$\\
\normalsize Versione classica \\
\footnotesize\textit{(simile a esistenza e unicità locale per EDO)}\\
\normalsize $$\big\Downarrow$$\\
\normalsize Versione invariante \\
\footnotesize\textit{(superfici non caratteristiche)}\\
\end{center}


Versione classica

\begin{theorem}
\vspace{-5mm}
\hpthml{
\overline{\mathcal{O}}_0 \subseteq \mathcal{O}_1 \subseteq \mathbb{C}^n \text{ aperti connessi limitati}\\
A_j, f, y_0 \text{ olomorfi in } z\\
A_j, f \text{ continui in } t\\
\text{Pb:}
\begin{cases}
D_t y = \sum A_j (z,t) D_{z_j}y+A_0(z,t)y +f(z,t) \\
y(z,0)=y_0(z)
\end{cases}\\
}{
\exists \, \delta \in (0,T) : \exists !\, y \text{ sol. per } |t|<T \\
- \text{ olomorfa in } z\\
- \; C^1 \text{ in } t \quad \quad \quad \rightarrow (\neq \text{Holmgren})
}
\end{theorem}


\newpage
\section{Teorema di Holmgren}

Risultato di \textbf{unicità} delle soluzioni per EDP lineari.
\begin{remark}
Il teorema di Cauchy-Kowalevski non esclude l'esistenza di altre soluzioni che non sono analitiche!
\end{remark}


\renewcommand{\arraystretch}{1.5}
\begin{tabular}{r||ccccc} 
CK & astratto & $\implies$  & classico & $\implies$ & invariante\\
&$\big\Downarrow$ &&&&\\
H & astratto & $\implies$ & classico & $\implies$ & invariante\\
\end{tabular}


Versione astratta:una qualsiasi equazione lineare può essere ridotta a un \textbf{sistema del $1$° ordine}. Ci concentriamo su questo caso. 
\begin{theorem}
\hpth{
\mathcal{O}_0= \{ z\in \mathbb{C}^n: |z|<r_0 \} \text{ con } r_0>0\\
y \text{ distribuzione su } (\mathcal{O}_0 \cap \mathbb{R}^n) \times (-T,T):\\
- K\subseteq  \mathcal{O}_0 \cap \mathbb{R}^n \text{ compatto: } y=0  \text{ in } \mathcal{O}_0 \cap \mathbb{R}^n \setminus K\\
- \begin{cases}
D_t y = \sum A_j (z,t) D_{z_j}y+A_0(z,t)y \\
y=0 \text{ per } t<0
\end{cases}\\
}{
y = 0 \text{ in } (\mathcal{O}_0 \cap \mathbb{R}^n) \times (-T,T) 
}
\end{theorem}

Versione classica
\begin{theorem}
\hpth{
\Omega \subseteq \mathbb{R}^n \text{ aperto}\\
A_j \text{ analitici}\\
y\in C^1 (\Omega \times (-T,T)): \\ 
\begin{cases}
D_t y = \sum A_j (x,t) D_{x_j}y+A_0(x,t)y \\
y=0 \text{ per } t=0
\end{cases} \\
}{
y = 0 \text{ in un intorno di } \Omega \times \{ 0\}
}
\end{theorem}


\begin{proof}
E' un'applicazione della versione astratta alla funzione $$\widetilde{y}(x,t) = H(t) \, y(x,t),$$ 
la quale soddisfa sempre un sistema della stessa tipologia.
\end{proof}

\newpage
\section{Altre applicazioni}

Teorema di Cartan-Kähler: un teorema molto importante in geometria differenziale:
\begin{itemize}
\item sull'integrabilità di \textbf{sistemi differenziali esterni} (\textit{exterior differential systems})
\item che si dimostra utilizzando il teorema di Cauchy-Kowalevski
\item che ha un'applicazione al campo economico (I. Ekeland, P.A. Chiappori)
\end{itemize}

Citando Ekeland a riguardo del paper scritto nel $1999$ insieme a Chiappori:\\
\begin{center}
\textit{Questo articolo risolve un problema di base nella teoria economica, che era rimasto aperto per \textbf{trent'anni}, ovvero la caratterizzazione delle funzioni di domanda di mercato. Il metodo di dimostrazione consiste nel ridurre il problema a un sistema di equazioni differenziali alle derivate parziali non lineari, per il quale si cercano soluzioni convesse. Questo viene riscritto come un sistema differenziale esterno e viene risolto mediante il teorema di Cartan-Kähler, insieme ad alcune manipolazioni algebriche per ottenere la \textbf{convessità}.}
\end{center}

Nonostante la ricerca condotta in quegli anni
\begin{itemize}
\item non fosse guidata da applicazioni immediate
\item portò a risultati \textbf{deludenti} rispetto alle aspettative di Cauchy e Weierstrass
\end{itemize}
ha avuto un impatto gigantesco grazie alla comprensione delle soluzioni di sistemi di EDP che ci ha permesso di raggiungere.





\newpage
\section{note}
differenza tra holmgren e ck rispetto alla tesi di $C^1$

teorema inutile nella pratica che però ha ispirato ricerche e scoperte future cruciali e utili (come accadde per i grafi di eulero)

applicazioni:
\begin{itemize}
\item fisica matematica (cosa succcede nella realtà quando si hanno soluzioni analitiche locali?)
\item teorema di holmgren
\item teorema di cartan-kahler (pag 137) utile in geometria differenziale (suggerisce Tao), economic theory (microeconomia, ekeland e chiappori, 1999) per l'eistenza locale di una funzione utilità concava (e quindi con un massimo) ricavata da un sistema di pdes che neccessita di dati analitici (visto come un sistema differenziale esterno)
\end{itemize}

I. Ekeland

Cartan Kahler

4 (with Chiappori), "Aggregation and market demand: an exterior differential calculus viewpoint", Econometrica, 67 (1999), p. 1435-1458

This paper solves a basic problem in economic theory, which had remained open for thirty years, namely the characterization of market demand functions. The method of proof consists of reducing the problem to a system of nonlinear PDEs, for which convex solutions are sought. This is rewritten as an exterior differential system, and is solved by the Cartan-Kähler theorem, together with some algebraic manipulations to achieve convexity. The introduction of exterior differential calculus proved to be a breakthrough, and was the starting point of a long collaboration with P.A. Chiappori. We realized that the mathematical structure we had discovered in this problem was to be found also in one of the major problems of econometrics: given a group (a household, for instance), can one characterize and identify the preferences of each member if one observes only the collective demand ?. I am happy to say that this research program is now concluded, with the publication of two major papers [14] and [25] and a100-pages survey [28] which will probably turn into a book.

14: (with P.A. Chiappori) "The microeconomics of group behaviour: general characterization". Journal of Economic Theory, september 2006 , volume 130 (p.1 – 26)

25: (with P.A. Chiappori) "The microeconomics of group behaviour: Identification". Econometrica, 2005, 44 pages

28: (with P.A. Chiappori). "The mathematics and economics of aggregation". Foundations and Trends in Economic Theory, 2009
