\chapter{Sviluppi successivi}

Questo risultato ha portato con sé alcuni sviluppi successivi in diversi campi durante il corso del Novecento. 
Prima di tutto, ricordiamo che, nonostante le aspettative di Cauchy, e poi di Weierstrass, il risultato di Kowalevski rivelò la natura intrinsecamente più complicata delle EDP. In particolare, contribuì a confutare definitivamente la congettura formulata da Weierstrass, che abbiamo già citato nel paragrafo \ref{introck}.

Le conseguenze più immediate, su cui ci concentreremo nei prossimi due paragrafi, corrispondono a due aspetti che riguardano proprio la teoria delle equazioni differenziali:
\begin{itemize}
\item le versioni alternative e più astratte del teorema;
\item il teorema di Holmgren, ovvero un risultato di esistenza e unicità (di una soluzione) per un sistema di EDP lineare nella classe di funzioni $C^1$.
\end{itemize}

Inoltre, tali conoscenze, stimolarono e ispirarono ulteriori approfondimenti. In particolare, i lavori che meritano di essere citati sono quelli
\begin{itemize}
\item di François Trèves e Louis Nirenberg, sulla ricerca di condizioni necessarie e/o sufficienti per l'esistenza di soluzioni locali in classi più ampie di funzioni;
\item di Lars Hörmander, su una teoria specifica per gli operatori differenziali lineari, grazie alla quale sono state trovate condizioni necessarie per l'esistenza e l'unicità delle soluzioni (per ulteriori dettagli si veda \cite{Hormander}).
\end{itemize}

\newpage
\section{Versioni alternative}

Nonostante il significato di fondo resti praticamente inalterato, se ci restringiamo al caso di un sistema lineare di EDP, gli enunciati possibili principali per il TCK sono tre e per distinguerli useremo rispettivamente tre aggettivi diversi per ciascuna versione: astratta, classica e invariante. In particolare, con l'ultimo termine ci riferiamo proprio alla versione che abbiamo trattato nel capitolo \ref{invariant}.

L'ordine in cui sono stati elencati i tre nomi non è casuale, infatti c'è una dipendenza logica tra questi enunciato che rappresentiamo con il seguente schema.

\begin{center}
Versione astratta $\implies$ Versione classica $\implies$ Versione invariante
\end{center}

Quindi, assumendo queste relazioni vere, possiamo dire che esiste un modo diverso per dimostrare il teorema che abbiamo già ampiamente discusso. 
Come suggeriscono i nomi, l'idea fondamentale dell'approccio è quella di analizzare il problema inserendolo in quadro teorico più generale e astratto, in modo da poter dedurre il teorema nella sua versione più comune (invariante) come un corollario.
Seguire questa via comporta prima di tutto un incremento sostanziale della difficoltà, e poi anche la perdita del legame diretto e immediato con l'idea di superficie caratteristica.

La nozione fondamentale da cui origina questa strada alternativa è quella delle classi di Ovsyannikov (ovvero insiemi di spazi di Banach composti da funzioni olomorfe), che vennero introdotte per la prima volta proprio dal matematico russo L. V. Ovsyannikov tra il 1960 e il 1970 (si veda \cite{Ovsyannikov}).

In questa tesi ci rifacciamo alla trattazione presente in \cite[cap.17-19]{Treves}, riportando però solo i passi salienti. Quindi, non ci concentriamo sulla costruzione della classi appena citate e non forniamo l'enunciato del teorema nella sua versione più astratta, ma ci soffermiamo solo sull'enunciato della versione classica, che permette già di cogliere e apprezzare nel concreto tutte le osservazioni fatte finora.

\begin{theorem}
\hpthml{
\overline{\mathcal{O}}_0 \subseteq \mathcal{O}_1 \subseteq \mathbb{C}^n \text{ aperti connessi limitati}\\
A_j, f, y_0 \text{ olomorfi in } z\\
A_j, f \text{ continui in } t\\
\text{Pb:}
\begin{cases}
D_t y = \sum A_j (z,t) D_{z_j}y+A_0(z,t)y +f(z,t) \\
y(z,0)=y_0(z)
\end{cases}\\
}{
\exists \, \delta \in (0,T) : \exists !\, y \text{ soluzione per } |t|<T \\
- \text{ olomorfa in } z\\
- \; C^1 \text{ in } t
}
\end{theorem}

\begin{remark}
Una qualsiasi equazione o sistema lineare può essere ridotta a un sistema del primo ordine, per questa ragione ci concentriamo solo su quest'ultimo caso.
\end{remark}

Non riportiamo la dimostrazione, in quanto è una semplice applicazione della versione più astratta, ma vogliamo capire come tale astrazione si rivela uno strumento utile a dimostrare il teorema di Holmgren.







\newpage
\section{Teorema di Holmgren}
Cominciamo ricordando che il risultato ottenuto da Kowalevski non fornisce alcuna informazione sull'esistenza di soluzioni non analitiche, le quali potrebbero, quindi, sia esistere che non esistere.
Di conseguenza, in questo paragrafo, vogliamo indagare cosa accade, sotto le ipotesi del TCK, quando ampliamo la classe di funzioni in cui cerchiamo le soluzioni di un problema. 
In particolare, ci chiadiamo: per quali equazioni e sotto quali ulteriori condizioni è garantita l'unicità della soluzione in una classe di funzioni più grande di quelle analitiche?

Una risposta interessante e generale a questa domanda è data proprio dal teorema di Holmgren, che, come accennato nel paragrafo precedente, può essere visto come conseguenza del TCK. 
La relazione logica tra questo teorema e la versione invariante del TCK, però, non è diretta; infatti, per la dimostrazione, è necessario appoggiarsi al quadro più astratto che abbiamo introdotto poco fa.

Per essere più precisi, come per il TCK, esistono tre versioni del teorema di Holmgren, che analogamente chiameremo: astratta, classica e invariante.
Quindi, integriamo lo schema già proposto in precedenza, aggiungendo il teorema di Holmgren.

\begin{center}
\renewcommand{\arraystretch}{1.5}
\begin{tabular}{r||ccccc} 
Cauchy-Kowalevski & astratto & $\implies$  & classico & $\implies$ & invariante\\
&$\big\Downarrow$ &&&&\\
Holmgren & astratto & $\implies$ & classico & $\implies$ & invariante\\
\end{tabular}
\end{center}

Questo risultato ci garantisce l'unicità della soluzione nella classe $C^1$, nel caso di equazioni lineari. Per sviscerare meglio il suo significato, vediamo per primo l'enunciato nella sua versione più astratta.

\begin{theorem}
\hpth{
\mathcal{O}_0= \{ z\in \mathbb{C}^n: |z|<r_0 \} \text{ con } r_0>0\\
y \text{ distribuzione su } (\mathcal{O}_0 \cap \mathbb{R}^n) \times (-T,T):\\
- K\subseteq  \mathcal{O}_0 \cap \mathbb{R}^n \text{ compatto: } y=0  \text{ in } \mathcal{O}_0 \cap \mathbb{R}^n \setminus K\\
- \begin{cases}
D_t y = \sum A_j (z,t) D_{z_j}y+A_0(z,t)y \\
y=0 \text{ per } t<0
\end{cases}\\
}{
y = 0 \text{ in } (\mathcal{O}_0 \cap \mathbb{R}^n) \times (-T,T) 
}
\end{theorem}

Ora, invece, enunciamo la versione classica e dimostriamola applicando quella astratta.

\begin{theorem}
\hpth{
\Omega \subseteq \mathbb{R}^n \text{ aperto}\\
A_j \text{ analitici}\\
y\in C^1 (\Omega \times (-T,T)): \\ 
\begin{cases}
D_t y = \sum A_j (x,t) D_{x_j}y+A_0(x,t)y \\
y=0 \text{ per } t=0
\end{cases} \\
}{
y = 0 \text{ in un intorno di } \Omega \times \{ 0\}
}
\end{theorem}

\begin{remark}
La differenza cruciale che si osserva tra questo enunciato e quello della versione classica del TCK risiede nel fatto che questa soluzione possa essere $C^1$ rispetto a tutte le variabili e non solo rispetto al tempo.
\end{remark}

\begin{proof}
E' un'applicazione della versione astratta alla funzione $$\widetilde{y}(x,t) = H(t) \, y(x,t),$$ 
la quale soddisfa sempre un sistema della stessa tipologia.
\end{proof}

\newpage
\section{Altre applicazioni}

Le conseguenze del lavoro di Kowalevski, però, non si fermano al campo delle equazioni differenziali, ma si possono trovare anche nei seguenti campi.

\begin{itemize}
\item Fisica matematica: per esempio, nel caso in cui si abbia un modello rappresentato da un sistema di EDP che soddisfa le ipotesi del TCK, ci si può chiedere se avere una soluzione locale analitica possa avere un significato fisico.
\item Geometria differenziale: grazie al TCK è stato possibile dimostrare il teorema di Cartan-Kähler sull'integrabilità di sistemi differenziali esterni (\textit{exterior differential systems}), che in spirito è molto simile al TCK.
\item Teoria economica: 

nel 1999
grazie al teorema di Cartan-Kähler è stato possibile dimostrare l'eistenza locale di una funzione utilità concava (e quindi con un massimo) ricavata da un sistema di EDP, il quale neccessità l'ipotesi di dati analitici (visto come un sistema differenziale esterno)

(I. Ekeland, P.A. Chiappori)
\end{itemize}

Ekeland riassume con queste parole il lavoro svolto insieme a Chiappori e che è riportato in \cite{CE}, \cite{CEgenchar}, \cite{CEaggregation} e \cite{CEid}:
\textit{This paper solves a basic problem in economic theory, which had remained open for \textbf{thirty years}, namely the characterization of market demand functions. The method of proof consists of reducing the problem to a system of nonlinear PDEs, for which convex solutions are sought. This is rewritten as an exterior differential system, and is solved by the Cartan-Kähler theorem, together with some algebraic manipulations to achieve \textbf{convexity}. The introduction of exterior differential calculus proved to be a breakthrough, and was the starting point of a long collaboration with P.A. Chiappori. We realized that the mathematical structure we had discovered in this problem was to be found also in one of the major problems of econometrics: given a group (a household, for instance), can one characterize and identify the preferences of each member if one observes only the collective demand? I am happy to say that this research program is now concluded [...].}

Traduzione in italiano:
\textit{Questo articolo risolve un problema fondamentale nella teoria economica, che era rimasto aperto per \textbf{trent'anni}, ossia la caratterizzazione delle funzioni di domanda di mercato. Il metodo di dimostrazione consiste nel ridurre il problema a un sistema di equazioni differenziali parziali non lineari, per il quale si cercano soluzioni convesse. Questo viene riscritto come un sistema differenziale esteriore, e viene risolto tramite il teorema di Cartan-Kähler, insieme ad alcune manipolazioni algebriche per ottenere la \textbf{convessità}. L'introduzione del calcolo differenziale esteriore si è rivelata una svolta e ha rappresentato il punto di partenza di una lunga collaborazione con P.A. Chiappori. Ci siamo resi conto che la struttura matematica che avevamo scoperto in questo problema era presente anche in uno dei principali problemi dell'econometria: dato un gruppo (per esempio una famiglia), è possibile caratterizzare e identificare le preferenze di ciascun membro osservando solo la domanda collettiva? Sono felice di dire che questo programma di ricerca è ora concluso [...].}

\newpage
\blankpage




